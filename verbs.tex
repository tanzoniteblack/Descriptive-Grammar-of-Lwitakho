\section{Verbs}

Verbs in Lwitaxo show agreement with the noun they work with.  The verbal agreement shows either the noun class of the subject or denotes the person which is the subject.  Lwitaxo shows the following persons in both the singular and plural: 1st, 2nd, 3rd.

There are 5 tenses in Lwitaxo: remote past, near past, present, near future, and far future.  There is also both perfective and imperfective forms of each of these tenses.

\subsection{Infinitives} \label{sec:infinitives}
Verb infinitives are nominal forms for the verb root, the nominal forms belong to noun class 15 with the prefix `xu--'.  The verb roots have a final vowel (FV), `--a', attached to them for syllabification purposes.\\
\fbox{xu + root + FV}

\subsection{Verb Types}
Short verbs behave differently than other verbs.  The easiest way to distinguish a short verb, is if the infinitive is only 2 syllables: xu\engma{}wa is a short root, due to the fact the root is only actually `\engma{}u' and this can be seen by the infinitive being 2 syllables.  When a short verb behaves differently in a form then a long verb, it will be noted.  

\subsection{Subject agreement}

The verb shows agreement with its subject through a prefix.  For most nouns the AGR2 (see section \ref{sec:agreement}) is used as the verb prefix, such as in the the class 6 word \emph{makondi} `sheep': \emph{makondi kalitsa\engma{}ga} `The sheep are eating.'

When dealing with human subjects, the agreement marker indicates whether the subject is 1st, 2nd, or 3rd person and whether the subject is singular or plural.  3rd person singular, the form corresponding with `he/she' in English, has two different forms depending on if the prefix is followed by a consonant: a--; or a vowel: j--.

The following table shows the prefixes affixed to the verb xuimba `to sing' in the present perfective form.  This is an example of the prefixes affixed to a verb with a root which begins with a vowel.  Note how `j--' is used for the 3rd person singular and how all the other forms besides 1st singular alternate to attach to the verb.

\noindent\begin{minipage}{\linewidth}
\small
\resizebox{\textwidth}{!}{ %
% Table generated by Excel2LaTeX from sheet 'Sheet2'
\noindent\begin{tabular}{rllllll}
\addlinespace
\toprule
    & \multicolumn{ 3}{l}{{\bf Sg}} & \multicolumn{ 3}{l}{{\bf Pl}} \\
\midrule
{\bf 1st} & n-- & `I' & \emph{nimb\'a} `I sing' & xu-- & `we' & \emph{xwimb\'a} `we sing' \\
{\bf 2nd} & u-- & `you' & \emph{wimb\'a} `you sing' & mu-- & `You' & \emph{mwimb\'a} `you all sing' \\
{\bf 3rd} & a--/j-- & `he/she' & \emph{jimb\'a} `he sings' & \beta{}a-- & `they' & \emph{\beta{}imb\'a} `they sing' \\
\bottomrule
\end{tabular}}
\normalsize
\end{minipage}

The following table shows the prefixes affixed to the verb \emph{xut\esh{}enda} `to walk' in the present perfective form.  This is an example of the prefixes affixed to a verb which has a root beginning with a consonant.  Note the change of [t\esh{}] to [d\ezh{}] due to the 1st singular prefix, as well as the use of `a--' for the 3rd singular form.


\noindent\begin{minipage}{\linewidth}
% Table generated by Excel2LaTeX from sheet 'Sheet2'
\resizebox{\textwidth}{!}{ %
\begin{tabular}{rlllllll}
\addlinespace
\toprule
      & \multicolumn{ 3}{l}{{\bf Sg}} & \multicolumn{ 3}{l}{{\bf Pl}} \\
\midrule
{\bf 1st} & n-- & `I' & \emph{ndʒenda} `I walk' & xu-- & `we' & \emph{xutʃenda} `we walk' \\
{\bf 2nd} & u-- & `you' & \emph{utʃenda} `you walk' & mu-- & `You' & \emph{mutʃenda} `you all walk' \\
{\bf 3rd} & a--/j-- & `he/she' &\emph{atʃenda} `he walks' & \beta{}a-- & `they' & \emph{\beta{}atʃenda} `they walk' \\
\bottomrule
\end{tabular}}
\normalsize
\end{minipage}

\subsubsection{First person singular}
The 1st person singular marker, n--, acts different than the other prefixes due to being a single consonant.  The other personal prefixes have vowels, which allow them to create a seperate syllable before consonant initial roots, but the 1st person singular requires special procedures:

\begin{description}

\item[Before a nasal] When conjugating a verb root that begins with a nasal, such as xuŋola `to find', the 1st person singular prefix is deleted: \emph{ŋola} `I find.'

\item[Before a voiceless fricative] When conjugating a verb root that begins with a voiceless fricative (f, s, \esh{}, x)\footnote{[h] does not phonologically act as a voiceless fricative in Lwitaxo}, such as xusoma `to read', then the 1st person singular prefix is deleted: \emph{soma} `I read.'

\item[Before `j'] If the verbal root begins with [j], such as xujixala `to sit', then the 1st person singular has 2 possible alternations: \emph{nixala}, where the [j] is deleted; and \emph{ndzixala} `I sit', where the [j] because [dz].  These two forms seem to be in free variation.

\item[All other cases] In all the remaining cases, the 1st person singular prefix affixes and causes the sound changes noted in the nasal section of the phonology section. 
\end{description}
\begin{wrdex}
\item \emph{xu\beta{}akala} `to spread out to dry' : mbakala `I spread (something) out to dry'
\item \emph{xut\'eha} `to draw water' : ndeha `I draw water'
\item \emph{xukona} `to sleep' : \engma{}gona `I sleep'.
\end{wrdex}

\subsubsection{Direct Object}
The direct object follows the verb.\\
\fbox{subject verb object}

\glll mwana asoma\engma{}ga \esh{}itapu 
subject verb object
child {is reading} book
\glt`The child is reading a book.'
\glend

\glll mwanafundzi jimba\engma{}ga luimbo
subject verb object
student {is singing} song
\glt `The student is singing a song'
\glend

If the direct object is used in a pronominal form, then agreement marker AGR2 for the direct object is put between the agreement marker for the subject and the verb.  The first person marker deletes, rather than combining with the consonant of the direct object marker.\\
\fbox{AGR2.subject+AGR2.object+verb}

\emph{ndol\'a\engma{}a} `I'm seeing', but \emph{lilol\'a\engma{}ga} `I'm seeing it (the stone, \emph{lit\esh{}ina}).'

\glll mwana a\emph{\textbf{\esh{}}i}soma\engma{}ga [\emph{\textbf{\esh{}i}}tapu]
subject {object+verb} [object]
child {is reading it} {[book]}
\glt`The child is reading it [a book].'
\glend

\glll mwanafundzi  a\emph{\textbf{lw}}imba\engma{}ga [\emph{\textbf{lu}}imbo]
subject {object+verb} object
student {is singing it} {[song]}
\glt `The student is singing it [a song]'
\glend

\subsection{Tense}

\subsubsection{A note on the imperfect}
The formation of the imperfective form, or progressive form, of each tense is shown below; however, with the exception of the remote past, all of the imperfective forms are predictable by adding the ending -V\engma{}gV to the perfective of the same tense, with something like vowel length possibly being added to the root, as in the near past form.  The final vowel in the verb form dictates whether the ending is \emph{-a\engma{}ga} or \emph{end\ezh{}e}.  If the verb form ends in [a], then the imperfective will take \emph{-a\engma{}ga}.  If the verb form ends in [i] or [e], then the imperfective will take \emph{-end\ezh{}e}.

\subsubsection{Remote Past}

The remote past is used when the action was more than approximately one day in the past.

\textbf{Perfect}: The perfective remote past tense marker is [a], which occurs after the agreement marker and before the root of the verb.\\
\fbox{AGR2 + a + root + FV}

\begin{wrdex}
\item \emph{xut\esh{}enda} `to walk' : \emph{n\textbf{a}tʃenda} `I walked'
\item \emph{xuŋgwa} `to drink' : \emph{w\textbf{a}ŋwa} `You drank'
\item \emph{xulia} `to eat' : \emph{j\textbf{\'a:}lja} `He ate'
\end{wrdex}

\textbf{Imperfect}
The imperfect of the remote past is a compound verb form made from the remote past of the verb to be, the connector \emph{ni}, and the present imperfective form of the verb.

% Table generated by Excel2LaTeX from sheet 'Sheet3'
\begin{tabular}{ll}
\addlinespace
\toprule
n\'a:li ni ndzil\'uxakaŋga & `I was running' \\
w\'a:li nu w\'{\i}mbaŋga & `You were singing' \\
j\'a:li na:litsaŋga & `You were eating' \\
xw\'a:li nu xuŋw\'etsanga & `We were drinking' \\
mw\'a:li nu muhuja\engma{}ga & `You all were going home' \\
\beta{}\'a:li na \beta{}ateha\engma{}ga & `They were drawing water' \\
\bottomrule
\end{tabular}

\subsubsection{Past Habitual}
The past habitual has the English translation of `I used to...but now I don't.'   It is formed by adding the imperfective marker `--a\engma{}ga' to the remote past perfective.\\
\fbox{AGR2 + a + root + a\engma{}ga}\\
or:\\
\fbox{Remote past + a\engma{}ga}

Short verbs add [ts] between the verb and the imperfective marker.

\begin{wrdex}
\item w+a+t\esh{}e:nd+a\engma{}ga: \emph{wat\esh{}e:nda\engma{}ga} `You used to walk, but now you don't'
\item  short verb: j+a:+l\'{\i}+ts+a\engma{}ga: \emph{ja:l\'{\i}tsa\engma{}ga} `He/she used to eat, but now they don't'
\item short verb: n+a+\engma{}we+ts+a\engma{}ga: \emph{na\engma{}wetsa\engma{}ga} `I used to drink, but now I don't'
\end{wrdex}

\subsubsection{Near Past}
The near past is used if the action was approximately within the last day.

\textbf{Perfect}: The perfective near past is formed by taking the root of a verb, applying the appropriate agreement marker, and then making the final vowel an [e].  The final vowel is affected by vowel harmony, see section \ref{sec:vowelharmony}.\\
\fbox{AGR2 + root + e}

Short verbs add -ele in this form.\\
\fbox{short verb: AGR2 + root+ ele}

\begin{wrdex}
\item n+dʒend+i : ndʒendi `I walked (recent)', from xut\esh{}enda
\item j+imb+i : jimbi `he sang (recent)', from xuimba
\item \engma{}w+ele : \engma{}wele `I drank (recent)', from xu\engma{}gwa
\end{wrdex}

\textbf{Imperfect}: The imperfective near past, also called the near past progressive form, is similar to the English ``I was doing."  The form is made by taking the perfective near past tense form and lengthening the final vowel of the verb root and adding the imperfective marker `--end\ezh{}e'.\\
\fbox{AGR2 + root(elongated vowel) + e + end\ezh{}e}\\
or\\
\fbox{Near Recent Past(elongated vowel) + end\ezh{}e}

\begin{wrdex}
\item n+dʒe:nd+i+end\ezh{}e : \emph{ndʒendiend\ezh{}e} `I walked (recent)', from \emph{xut\esh{}enda}
\item j+i:mb+i+end\ezh{}e : \emph{ji:mbiend\ezh{}e} `he sang (recent)', from \emph{xuimba}
\item \engma{}w+ele+end\ezh{}e : \emph{\engma{}we:le:nd\ezh{}e} `I drank (recent)', from \emph{xu\engma{}gwa}
\end{wrdex}

There is also a second form for the near past imperfective.  It is in fact identical to the form above, except with the \emph{-nd\ezh{}e}.  This makes the form near identical to the perfect near past forms which in [e], except for the lengthened. vowel in the root

\begin{wrdex}
\item n+dʒe:nd+i+e : \emph{ndʒendie} `I walked (recent)', from \emph{xut\esh{}enda}
\item j+i:mb+i+e : \emph{ji:mbie} `he sang (recent)', from \emph{xuimba}
\item \engma{}w+ele+e : \emph{\engma{}we:le} `I drank (recent)', from \emph{xu\engma{}gwa}
\end{wrdex}

\subsubsection{Present}
\textbf{Perfective}: The present perfective tense is used to discuss current habitual actions such as, ``I play" in the sense of ``I play with the ball."  The form is created by affixing AGR2 to the verb root and keeping the final vowel [a].\\
\fbox{AGR2 + root + FV}

Short verbs add [ts] to the end of the root to create the present tense.

\begin{wrdex}
\item short verb: ka+li+tsa : \emph{makondi kalitsa lusese} `Sheep eat grass', from \emph{xulia} `to eat'
\item n+d\'ex+a : \emph{nd\'exa m\'atsi} `I boil water', from \emph{xut\'exa} `to boil'
\end{wrdex}

\textbf{Imperfective}: The present imperfective tense is used as the present progressive form such, ``I am playing."  The form is made by adding the imperfective marker `--a\engma{}ga' to the present perfective form.\\
\fbox{AGR2 + root+ a\engma{}ga}\\
or\\
\fbox{Present perfective + a\engma{}ga}

\begin{wrdex}
\item \engma{}+gul+a\engma{}ga : \emph{\engma{}gula\engma{}ga} `I am buying', from \emph{xukula} `to buy'
\item a+\beta{}\'a:j+a\engma{}ga : \emph{a\beta{}\'a:ja\engma{}ga} `he is playing', from \emph{xu\beta{}a:ja} `to play'
\item a+si\`e\beta{}+a\engma{}ga : \emph{asi\'e\beta{}a\engma{}ga} `She/he is dancing', from \emph{xusie\beta{}a} `to dance'
\item short verb: \engma{}w\'e+ts+a\engma{}ga : \emph{\engma{}w\'etsa\engma{}ga} `I am drinking' from \emph{xu\engma{}wa} `to drink' 
\end{wrdex}

\subsubsection{Near Future}
The near future shows actions which are expected to happen soon, normally within the next few days.

\textbf{Perfective}: The near future perfective is used when the action being spoken about is in the near future and focus is being placed on the completion of the action.  The form is made by inserting the marker \emph{la} between the agreement marker and the root of the verb and the final vowel.  The \emph{la} is subject to the phonological rules: when the first person is used the l deletes, and only the a is visible; when the \emph{la} attaches to a verb root that begins with a vowel, the normal rules for [a] occur, see section \ref{sec:arules} for more details.\\
\fbox{AGR2 + la + root + FV}

\begin{wrdex}
\item u+la+huj+a : \emph{ulahuja} `You will go home', from \emph{xuhuja} `to go home'
\item n+la+hamb+a : \emph{nahamba} `I will catch [something] on fire', from \emph{xuhamba} `to catch on fire'
\item a+la+iluxak+a : \emph{aliluxaka} `He/she will run', from \emph{xwiluxaka} `to run'
\item \beta{}a+la+kon+a : \emph{\beta{}alakona} `They will sleep' from \emph{xukona} `to sleep'
\end{wrdex}

\textbf{Imperfective}: The near future imperfective is used when the action being spoken about is in the near future and the focus is on the process of doing it.  This form coincides with English's `will be doing.'  The form is made by adding the imperfective marker \emph{-a\engma{}ga} to the perfective near future.  Note that the final vowel of the perfective form is assimilated by the imperfective marker.\\
\fbox{AGR2 + la + root + a\engma{}ga}\\
or\\
\fbox{Near future perfective + a\engma{}ga}

Short verbs once again add the [ts] after the root to create this form.

\begin{wrdex}
\item a+la+lila+a\engma{}ga : \emph{alalila\engma{}ga} `He/she will be crying', from \emph{xulila} `to cry'
\item n+la+mal+a\engma{}ga : \emph{namala\engma{}a} `I will be finishing', from \emph{xumala} `to finish'
\item u+la+sax+a\engma{}ga:\emph{ulasaxa\engma{}ga} `You will be laughing', from \emph{xusaxa} `to laugh'
\item short verb: n+la+\engma{}wa+ts+a\engma{}ga : \emph{na\engma{}watsa\engma{}ga} `I will be drinking', from \emph{xu\engma{}wa} `to drink'
\end{wrdex}

\subsubsection{Remote Future}
The remote future, or far future, is used when an action might take place, but when it does it will be in several weeks.  It is referred to by the informant as an almost `prophetic' or `wishful thinking' form.

\textbf{Perfective}: The perfective remote future is used when the action will possibly take place in the distant future, and the focus is being placed on the completion of the action.  The form is created inserting the marker \emph{axa} between the agreement marker and the root of the verb\footnote{normal rules for [a] occur, see section \ref{sec:arules} for more details}, and changing the final vowel to [e] (vowel harmony rules apply with high vowels making this vowel [i]).\\
\fbox{AGR2 + axa + root + e}

\begin{wrdex}
\item n+axa+\engma{}w+i : \emph{naxa\engma{}wi} `I will drink (distant future)', from \emph{xu\engma{}gwa}
\item j+axa+\beta{}a:j+e : \emph{jaxa\beta{}a:je} `He will play (distant future)', from \emph{xu\beta{}a:ja}
\item n+axa+t\esh{}end+e : \emph{naxat\esh{}end\ezh{}e} ` I will walk (distant future)', from \emph{xut\esh{}enda}
\end{wrdex}

\textbf{Imperfective}: The imperfective remote future is used when the action will possibly take place in the distant future, and the focus is being placed on the process of doing the action.  The form is created by adding the imperfective form \emph{-end\ezh{}e} to the perfective far future.\\
\fbox{AGR2 + axa + root + end\ezh{}e}\\
or\\
\fbox{Far future perfective + end\ezh{}e}

The short verbs insert [ts] after the root.
\begin{wrdex}
\item \beta{}a+ax\'akul+end\ezh{}e : \emph{\beta{}ax\'akulend\ezh{}e} `They will be buying,' infinitive: \emph{xukola} `to buy'
\item u+axa+iluxak+end\ezh{}e : \emph{waxiluxat\esh{}end\ezh{}e} `You will be running,' infinitive: \emph{xwiluxaka} `to run'
\item short verb: n+axa+\engma{}we+ts+end\ezh{}e : \emph{naxa\engma{}wetsend\ezh{}e} `I will be drinking,' infinitive: \emph{xu\engma{}wa} `to drink'
\end{wrdex}

\subsection{Negation of Verbs}
Verbs are negated by adding the prefix \emph{\esh{}i--} before a conjugated verb, and the negation particle \emph{t\'a:we} after the verb.\\
\fbox{\esh{}i+verb + t\'a:we}

\begin{wrdex}
\item \esh{}i+w\'a:li nu w\'{\i}mbaŋga+t\'awe : \emph{\esh{}iw\'a:li nu w\'{\i}mbaŋga t\'awe} `You were not walking (remote past)
\item \esh{}i+xutʃenda+t\'a:we : \emph{\esh{}ixutʃenda t\'awe} `We do not walk'
\item \esh{}i+jaxa\beta{}a:je+t\'a:we : \emph{\esh{}ijaxa\beta{}a:je t\'a:we} `He will not play (remote future)
\item \esh{}i+ naxa\engma{}wetsend\ezh{}e+t\'awe : \emph{\esh{}inaxa\engma{}wetsend\ezh{}e t\'a:we} `I will not be drinking (remote future)'
\end{wrdex}

\subsection{Imperatives}
There are two imperatives: one for speaking to a single subject, and one for speaking to multiple subjects at the same time.  The singular imperative is formed by the verb root plus [a].  The plural imperative is formed by the verb root plus [i].\\
\fbox{Singular imperative: root+a}\\
\fbox{Plural imperative: root+i}

\begin{wrdex}
\item \emph{jixala} `sit!' to sg, \emph{jixali} `sit!' to pl, infinitive: \emph{xujixala} `to sit'
\item \emph{hamba} `come!' to sg, \emph{hambi} `come!' to pl, verb root : \emph{hamb} (different root then the verb to come normally uses)
\item \emph{kona} `sleep!' to sg, \emph{koni} `sleep!' to pl, infinitive: \emph{xukona}
\end{wrdex}

\subsubsection{Negative Imperatives}
The negative imperative, the form equivalent to "Don't do..." in English, also has different forms depending on if it is being said to a single subject or multiple subjects at the same time.  The singular is created by placing \emph{uxa} before the positive singular imperative, and the negation particle \emph{t\'a:we} after.  The plural is created in the same way, but with \emph{muxa} before the positive plural imperative instead.\\
\fbox{singular: uxa + imperative sg. + t\'a:we}\\
\fbox{plural: muxa + imperative pl. + t\'awe}

\begin{wrdex}
\item \emph{uxa jixala t\'a:we} `don't sit!' to sg, \emph{muxa jixali t\'a:we} `don't sit!' to pl, infinitive: \emph{xujixala} `to sit'
\item \emph{uxa hamba t\'a:we} `don't come!' to sg, \emph{muxa hambi t\'a:we} `don't come!' to pl, verb root : \emph{hamb} (different root then the verb to come normally uses)
\item \emph{uxa kona t\'a:we} `don't sleep!' to sg, \emph{muxa koni t\'a:we} `don't sleep!' to pl, infinitive: \emph{xukona}
\end{wrdex}

\subsection{Benefactive Mood}
The benefactive mood, to do something for the benefit of someone else, is created by inserting the marker \emph{il} after the root of the verb.  This marker is subject to vowel harmony: if the preceeding vowel is a midvowel, then the marker changes to \emph{el}.\\
\fbox{AGR2 + root + BEN + (impf) + FV}
\begin{wrdex}
\item j+aj+il+a\engma{}ga : \emph{jajila\engma{}ga} `he was grazing [animals] for...'
\item n+a+teh+el+a\engma{}ga : 
\gll natehela\engma{}ga kuka wand\ezh{}e ma:tsi
{I used to draw} grandfather my water
\glt `I used to draw water for my grandfather'
\glend
\end{wrdex}

\subsection{Compound Forms}
There are a few verb forms in Lwitaxo which are compound verb phrases, where two verbs are used in conjunction to get a particular form.  One of these is the form which translates to in English, `will have just', in the sense `They will have just finished something, or ate something.'  This form is created by using the future perfect form of the verb \emph{xu\beta{}a}, either near future or distant future depending on how far in the future in the action will have been completed, and the action which will have been completed in the distant future perfective.  There optionally can be placed the connective word \emph{ni} between the two verbs.
\begin{wrdex}
\item \emph{\beta{}ala\beta{}\'a \beta{}\'axamala}  `They will have just finished (near future)'
\item \emph{\beta{}axa\beta{}\'e (ni) \beta{}\'axamala} `They will have just finished (distant future)'
\item \emph{na\beta{}a nimali xusoma \esh{}itapu} `I will have finishsed reading the book'
\end{wrdex}

\subsection{xu\beta{}a `to be'}
Due to the verb to be being slightly irregular and very common, here is a reference chart of its conjugations.\\
% Table generated by Excel2LaTeX from sheet 'Sheet4'
\resizebox{\textwidth}{!}{ %   
 \begin{tabular}{llllll}
    \addlinespace
    \toprule
    \multicolumn{ 6}{c}{{\bf xuβa `to be'}}       \\
    \midrule
    {\bf Person} & {\bf remote past} & {\bf near past} & {\bf present} & {\bf near future} & {\bf remote future} \\
    1sg   & na:li & mbele & ndi   & naβa  & na:xaβe \\
    2sg   & wa:li & uβele & uli   & ulaβa & waxaβe \\
    3sg   & ja:li & aβele & ali   & alaβa & jaxaβe \\
    1pl   & xwa:li & xuβele & xuli  & xulaβa & xuxaβe \\
    2pl   & mwa:li & muβele & muli  & mulaβa & muxaβe \\
    3pl   & βa:li & βaβele & βuli  & βalaβa & βaxaβe \\
    \bottomrule
    \end{tabular}}