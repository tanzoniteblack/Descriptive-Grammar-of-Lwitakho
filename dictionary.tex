\begin{description}

\item[xuxu\beta{}ana] khukhubana; `to beat each other', `to exchange blows'.  This verb is used in the contexts where a fight is between individuals who are not trying to do fatal harm to each other (where the verb xukwana might be used instead), but are fighting in the sense of exchanging blows.  This is used in contexts such as fighting sports like boxing, or `play fights'.
\item[xulembana] khulembana; `to quarrel' unlike \emph{xuhi\engma{}gana} `to argue', this veb saya that not only are the participants disagreeing about something, but they are doing such in raised, angry tones.  This kind of an argument has the potential to break out into a fight.
\item[xusoma] khusoma; `to read' or `to study'; used to indicate the action of spending some time trying to understand material which is in a written form.  Can be used both for pleasure or for academic interests.
\item[i\engma{}ombe] `cow', this is a generic noun that represents any kind of cow; it can be subdivided into the following words: \emph{ihunwa} `male cow, steer'; \emph{it\esh{}ili\esh{}i} `bull (male cow for breeding)'; \emph{imosi} `female cow'; and \emph{itwasi} `female cow that has given birth'
\item[lit\esh{}ina,mat\esh{}ina] `stone'; used not only for stones, rocks, and boulders, but also for anything non-organic and hard in this general shape such as hail.  \emph{imbula ja mat\esh{}ina} `rain with rocks in it' or `rain with hail'
\item[lwaxo, tsinzaxo] `fence', `boundary between lands'; used for any sort of division between regions of land.  A fence, the line where two homesteads meet, the border between two countries, etc
\item[-pi] `bad'; this can be used in either the sense of `bad look' \emph{likono lipi} or a `bad person' \emph{mundu upi}, also if something tastes or smells bad; in general just shows a negative quality;  a more extreme form is \emph{-tamanu} 

\end{description}