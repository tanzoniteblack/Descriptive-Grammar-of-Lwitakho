
\subsection{Agreement Patterns} \label{sec:agreement}
All modifiers of nouns, as well as verbs, have agreement with the noun.  The agreement is marked by an agreement marker, normally taking the form of a prefix.  There are three different agreement markers:
\begin{inparaenum}[\itshape a\upshape)]
%\renewcommand{\labelenumi}{\alph{enumi})}
\item \label{agrdem} The agreement marker used for demonstratives, abbreviated as DEM;
\item \label{agr1} AGR1, used for adjectives, and the number one; and
\item \label{agr2} AGR2, used for possessives, linking words, verbs, and numbers two through five.
\end{inparaenum}  The only classes where AGR2 is distinctive from AGR1 are classes 3, 4 and 6.  When the agreement marker is prefixed to a stem that starts with a vowel the vowel from the agreement marker is deleted, unless it is a high vowel and then it creates a glide.

The locative classes do not have agreement markers for nouns that are being used for a locative meaning but rather use the agreement marker for whatever class the noun normally is; i.e. words like \emph{ha:si} `floor' which are innately one of the locative classes will have the agreement marker \emph{ha-}, but a word like \emph{halitu:ka} `near the shop' would use the agreement marker for class class 5, as shop is \emph{litu:ka} and a class 5 noun.

Examples:\begin{wrdex}
\item \emph{\textbf{ma}kondi \textbf{ka}nd\ezh{}e} `my sheep' (class 6) 
\item \emph{\textbf{li}xoxo \textbf{lj}e\textbf{lj}e} `his/her turkey' (class 5)
\item \emph{\textbf{mu}koje \textbf{kw}aβo} `their catfish' (class 3)
\item \emph{\textbf{β}etʃitsi \textbf{β}o\textbf{β}o} `your (sg) teachers' (class 2)
\end{wrdex} 

\small
% Table generated by Excel2LaTeX from sheet 'agr'
\begin{tabular}{rllll}
\addlinespace
\toprule
\multicolumn{ 5}{c}{{\bf Agreement Marker Chart}} \\
\midrule
{\bf Class \# } & {\bf  Prefix } & {\bf  DEM } & {\bf  AGR1 } & {\bf  AGR2 } \\
\midrule
1     &  mu   &  uj   &  u    &  u \\
2     &  \beta{}a  &  a\beta{}  &  \beta{}a  &  \beta{}a \\
3     &  mu   &  uk   &  mu   &  \bf{ku}  \\
4     &  mi   &  it\esh{}  &  mi   &  \bf{it\esh{}}  \\
5     &  li   &  il   &  li   &  li \\
6     &  ma   &  ak   &  ma   &  \bf{ka}  \\
7     &  \esh{}i  &  i\esh{}  &  \esh{}i  &  \esh{}i  \\
8     &  \beta{}i  &  i\beta{}  &  \beta{}i  &  \beta{}i  \\
9     &  i    &  ij   &  i    &  i  \\
10    &  tsi  &  its  &  tsi  &  tsi \\
11    &  lu   &  ul   &  lu   &  lu \\
12    &  xa   &  ax   &  xa   &  xa \\
13    &  ru   &  ur   &  ru   &  ru \\
14    &  \beta{}u  &  u\beta{}  &  \beta{}u  &  \beta{}u \\
15    &  xu   &  ux   &  xu   &  xu \\
\bottomrule
\end{tabular}
\normalsize

\subsection{Adjectives}
Most adjectives use the same process to show agreement to their head of the noun phrase: AGR1+adjective root.  The adjective follows the noun in the noun phrase.

% Table generated by Excel2LaTeX from sheet 'Sheet1'
\begin{tabular}{rlll}
\addlinespace
\toprule
      & {\bf AGR1} & {\bf  -kali `big'} & {\bf  -laji `good'} \\
\midrule
{\bf Class 1} & u     & ukali & ulaji \\
{\bf Class 2} & \beta{}a & \beta{}akali & \beta{}alaji \\
{\bf Class 3} & mu    & mukali & mulaji \\
{\bf Class 4} & mi    & mikali & milaji \\
{\bf Class 5} & li    & likali & lilaji \\
{\bf Class 6} & ma    & makali & malaji \\
{\bf Class 7} & \esh{}i & \esh{}ikali & \esh{}ilaji \\
{\bf Class 8} & \beta{}i & \beta{}ikali & \beta{}ilaji \\
{\bf Class 9} & iN     & i\engma{}gali & ilaji \\
{\bf Class 10} & tsiN   & tsi\engma{}gali & tsilaji \\
{\bf Class 11} & lu    & lukali & lulaji \\
\bottomrule
\end{tabular}

\subsubsection{Adjectives type 2}
There is another type of adjective as well, which forms in the same way except uses AGR2, rather than AGR1 to show agreement.  The exact reason why some adjectives use AGR1 vs. AGR2 is unclear at this time.  However, only the classes with nasal initial prefixes (classes 3, 4 and 6) are affected by this different type of adjective.  Example: `hot' is kuhili (class 3), t\esh{}ihili (class 4), and kahilii (class 6).

\subsection{Numbers}
The number 1 shows agreement with a noun it modifies by using the form \fbox{AGR1+number}.  Numbers 2-5 agree with the noun they modify by using the form \fbox{AGR2+number}.  Numbers 6-10 are set in their form, and do not show agreement with the noun they modify.  When used for counting, or in general when not used with a noun, the number 1 has the prefix `lu-' (class 11 agreement), the numbers 2-5 have the prefix `xa-' (class 12 agreement).  Numbers occur after the noun they modify.

\noindent\begin{minipage}{\linewidth}
\resizebox{\textwidth}{!}{ %
% Table generated by Excel2LaTeX from sheet 'Sheet1'
\begin{tabular}{rlllll}
\addlinespace
\toprule
      &       &       & \multicolumn{ 3}{l}{Example of agreement with:} \\
\midrule
     \# & \emph{ root} & \emph{ Counting} & \emph{ mundu `person'} & \emph{ isa `clock'} & \emph{ likondi `sheep'} \\
{\bf 1} & -lala & lulala & mulala & indala & lilala \\
{\bf 2} & -βili & xa\beta{}ili & βaβili & tsiβili & maβili \\
{\bf 3} & -βaka & xa\beta{}aka & βaβaka & tsiβaka & maβaka \\
{\bf 4} & -xane & xane  & βaxane & tsixane & maxane \\
{\bf 5} & -ranu & xaranu & βaranu & tsiranu & maranu \\
{\bf 6} & sita  & sita  & sita  & sita  & sita \\
{\bf 7} & sapa  & sapa  & sapa  & sapa  & sapa \\
{\bf 8} & munane & munane & munane & munane & munane \\
{\bf 9} & tisa  & tisa  & tisa  & tisa  & tisa \\
{\bf 10} & lixomi & lixomi & lixomi & lixomi & lixomi \\
\bottomrule
\end{tabular}}
\normalsize
\end{minipage}

\subsection{Possessive Adjectives} \label{sec:possessives}
The Lwitaxo language only allows for humans to posses things, therefore there are only possessive adjectives for humans and not for any other noun class.  There are six types of possessive adjectives, a singular and plural for each 1st, 2nd, and 3rd persons.  Each of these six possessive adjectives has to agree to the noun that is being possessed.  The specific forms for each person are described below, and the agreement marker shows agreement with the noun being possessed.  The possessive adjective follows the noun is modifies, for more complex noun phrases refer back to section \ref{sec:nounwordorder} and for the agreement chart refer back to section \ref{sec:agreement}.

\subsubsection{First person singular, `my'}
The possessive adjective for first person singular, translated into English as `my' is formed:

\fbox{AGR2 + and\ezh{}e}

Remember that AGR2 here refers to the agreement marker for the noun that is being possessed.

Examples:\begin{wrdex}
\item `my book' (class 7) : \esh{}itapu \esh{}j+and\ezh{}e : \emph{\esh{}itapu \esh{}jand\ezh{}e}
\item `my  sheep (plural)' (class 6) : makondi k+and\ezh{}e : \emph{makondi kand\ezh{}e}
\end{wrdex}

\subsubsection{Second person singular, `your'}
The possessive adjective for second person singular, or `your', is formed

\fbox{AGR2+o+AGR2+o}

The agreement marker is used twice in this formation.

Examples:\begin{wrdex}
\item `your cow' (class 9) : i\engma{}ombe j+o+j+o : \emph{i\engma{}ombe jojo}
\item `your mother' (class 1a) : mama w+o+w+o : \emph{mama wowo}\footnote{class 1a does not have a prefix, but is treated as class 1}
\item`your trees' (class 4) : misala t\esh{}j+o+t\esh{}j+o : \emph{misala t\esh{}jot\esh{}jo}
\end{wrdex}

\subsubsection{Third person singular, `his/her'}
Lwitaxo does not have a distinction between masculine and feminine, and so there is only one third person pronoun which can be translated either as `his' or `her'.  It is formed by:

\fbox{AGR2+e+AGR2+e}

Again the agreement marker, AGR2, is used twice in the formation of the possessive.

Examples:\begin{wrdex}
\item `his/her chicken' (class 5) : lixoxo lj+e+lj+e :  \emph{lixoxo ljelje}
\item `his/her brother' (class 1a) : mbotso w+e+w+e : \emph{mbotso wewe}
\item `his/her funeral' (class 6) : malika k+e+k+e : \emph{malika keke}
\end{wrdex}

\subsubsection{First person plural, `our'}
The plural first person possessive adjective, or `our', is formed:

\fbox{AGR2+eru}

Examples:\begin{wrdex}
\item `our fence' (class 11) : lukaka lw+eru : \emph{lukaka lweru}
\item `our house' (class 9): indzu j+eru: \emph{indzu jeru}
\item `our sweet potato vines' (class8) : βipwoni \beta{}j+eru : \emph{βipwoni \beta{}jeru}
\end{wrdex}

\subsubsection{Second person plural, `your'}
The second person plural possessive adjective, or `your' when talking to a group of people, is formed:

\fbox{AGR2 + e\palnas{}u}

Examples:
\begin{wrdex}
\item `your (pl\footnote{plural}) yams' (class 10) : \emph{tsinudma tsj+e\palnas{}u : tsinudma tsje\palnas{}u}
\item `your (pl) bananas' (class 6) : maramwa k+e\palnas{}u : \emph{maramwa ke\palnas{}u}
\item `your (pl) childhood' (class 14) : βuana \beta{}w+e\palnas{}u : \emph{βuana \beta{}we\palnas{}u}
\end{wrdex}

\subsubsection{Third person plural, `their'}
The third person plural possessive adjective, or `their', is formed by:

\fbox{AGR2+a\beta{}o}

Examples:\begin{wrdex}
\item `their boat' (class 5) : ljaro lj+a\beta{}o : \emph{ljaro lja\beta{}o}
\item `their room' (class 9) : irumu j+a\beta{}o: \emph{irumu ja\beta{}o}
\item `their teacher' (class 1) : mwetʃitsi w+a\beta{}o : \emph{mwetʃitsi wa\beta{}o}
\item `their dreams' (class 6) : maloro k+a\beta{}o : \emph{maloro ka\beta{}o}
\end{wrdex}

\noindent\begin{minipage}{\linewidth}
\resizebox{\textwidth}{!}{ %
% Table generated by Excel2LaTeX from sheet 'posadj'
\begin{tabular}{rlllllll}
\addlinespace
\toprule
\multicolumn{ 8}{c}{{\bf Possessive Adjectives}}              \\
\midrule
{\bf Class} & {\bf  AGR2 } & {\bf  `my' } & {\bf  `your' } & {\bf  `his/her' } & {\bf  `our' } & {\bf  `your (pl.)' } & {\bf  `their' } \\
1     &  u    &  wand\ezh{}e  &  wowo  &  wewe  &  weru  &  we\palnas{}  &  wa\beta{}o  \\
2     &  \beta{}a  &  \beta{}and\ezh{}e  &  \beta{}o\beta{}o  &  \beta{}e\beta{}e  &  \beta{}eru  &  \beta{}e\palnas{}  &  \beta{}a\beta{}o  \\
3     &  ku   &  kwand\ezh{}e  &  kwokwo  &  kwekwe  &  kweru  &  kwe\palnas{}  &  kwa\beta{}o  \\
4     &  t\esh{}i  &  t\esh{}jand\ezh{}e  &  t\esh{}jot\esh{}jo  &  t\esh{}jet\esh{}je  &  t\esh{}jeru  &  t\esh{}je\palnas{}u  &  t\esh{}ja\beta{}o  \\
5     &  li   &  ljand\ezh{}e  &  ljoljo  &  ljelje  &  ljeru  &  lje\palnas{}u  &  lja\beta{}o  \\
6     &  ka   &  kand\ezh{}e  &  koko  &  keke  &  keru  &  ke\palnas{}u  &  ka\beta{}o  \\
7     &  \esh{}i  &  \esh{}jand\ezh{}e  &  \esh{}jo\esh{}jo  &  \esh{}je\esh{}je  &  \esh{}jeru  &  \esh{}je\palnas{}u  &  \esh{}ja\beta{}o  \\
8     &  \beta{}i  &  \beta{}jand\ezh{}e  &  \beta{}jo\beta{}jo  &  \beta{}je\beta{}je  &  \beta{}jeru  &  \beta{}je\palnas{}u  &  \beta{}ja\beta{}o \\
9     &  ji   &  jand\ezh{}e  &  jojo  &  jeje  &  jeru  &  je\palnas{}u  &  ja\beta{}o \\
10    &  tsi  &  tsjand\ezh{}e  &  tsjotsjo  &  tsjetsje  &  tsjeru  &  tsje\palnas{}u  &  tsja\beta{}o  \\
11    &  lu   &  lwand\ezh{}e  &  lwolwo  &  lwelwe  &  lweru  &  lwe\palnas{}u  &  lwa\beta{}o  \\
12    &  xa   &  xand\ezh{}e  &  xoxo  &  xexe  &  xeru  &  xe\palnas{}u  &  xa\beta{}o  \\
13    &  ru   &  rwand\ezh{}e  &  rworwo  &  rwerwe  &  rweru  &  rwe\palnas{}u  &  rwa\beta{}o  \\
14    &  \beta{}u  &  \beta{}wand\ezh{}e  &  \beta{}wo\beta{}wo  &  \beta{}we\beta{}we  &  \beta{}weru  &  \beta{}we\palnas{}u  &  \beta{}wa\beta{}o  \\
15    &  xu   &  xwand\ezh{}e  &  xwoxwo  &  xwexwe  &  xweru  &  xwe\palnas{}u  &  xwa\beta{}o  \\
\bottomrule
\end{tabular}}
\normalsize
\end{minipage}

\subsection{Demonstratives} \label{sec:demon}
Lwitaxo appears to have two different demonstrative adjectives: one that denotes proximity to the speaker, and another that shows distance from the speaker.  Many Bantu languages have a three way distinction: one denoting proximity to the speaker, another for proximity to listener, and a third for distant from both listener and speaker.  Demonstratives come after the noun they modify, for more complex noun phrases refer back to section \ref{sec:nounwordorder}.

\subsubsection{Proximal, `this/these'}
The demonstrative adjective which is translated into English as either `this' or `these' is formed by:

\fbox{j + DEM + V}

DEM is a form of agreement marker.  The agreement markers are listed in section \ref{sec:agreement}, and also is relisted in the table below.  The V in the proximal is the same as the vowel in the DEM marker.

Examples:\begin{wrdex}
\item `this sheep' (class 5) : likondi j+il+i : \emph{likondi jili}
\item `these hawks' (class 8) : \beta{}ilitsa j+i\beta{}+i : \emph{\beta{}ilitsa ji\beta{}i}
\end{wrdex}

An exception to this is the form of the demonstrative for class 1 nouns.  The `j' does not appear in the beginning of the demonstrative; `this person' is \emph{mundu uju}, where \emph{uju} is the demonstrative.

\subsubsection{Distal, `that/those'}
The demonstrative adjective translated into English as `that' and `those' is formed by:

\fbox{j + DEM + o}

The DEM agreement marker can be found either in section \ref{sec:agreement} or in the table below.  The `o' has the effect of lowering high vowels (i and u) in the DEM marker to `e' and `o'.

Examples:\begin{wrdex}
\item `that horn' (class 11) : luika j+ol+o : \emph{luika jolo}
\item `that fish' (class 9) : isutse j+ej+o : \emph{isutse jejo}
\item `that hare' (class 12) : xamuna j+ax+o : \emph{xamuna jaxo}
\end{wrdex}

The class 1 form for the demonstrative once again follows the the rule of not having the initial `j': `that person' is \emph{mundu ojo}, where \emph{ojo} is the demonstrative.

% Table generated by Excel2LaTeX from sheet 'demadj'
\small
\noindent \begin{tabular}{rrrr}
\addlinespace
\toprule
\multicolumn{ 4}{c}{{\bf Demonstrative Adjective Chart}} \\
\midrule
{\bf Class \# } & {\bf  DEM } & {\bf  Proximal } & {\bf  Distal } \\
\midrule
1     &  uj   &  uju  &  ojo \\
2     &  a\beta{}  &  ja\beta{}a  &  ja\beta{}o \\
3     &  uk   &  juku  &  joko  \\
4     &  it\esh{}  &  jit\esh{}i  &  jet\esh{}o \\
5     &  il   &  jili  &  jelo \\
6     &  ak   &  jaka  &  jako  \\
7     &  i\esh{}  &  ji\esh{}i  &  je\esh{}o  \\
8     &  i\beta{}  &  ji\beta{}i  &  je\beta{}o  \\
9     &  ij   &  jiji  &  jejo  \\
10    &  its  &  jitsi  &  jetso \\
11    &  ul   &  julu  &  jolo \\
12    &  ax   &  jaxa  &  jaxo \\
13    &  ur   &  juru  &  joro \\
14    &  u\beta{}  &  ju\beta{}u  &  jo\beta{}o \\
15    &  ux   &  juxu  &  joxo \\
\bottomrule
\end{tabular}
\normalsize