\section{Nouns}
Nouns in Lwitaxo almost always consist of two segments (morphemes): a noun prefix (NP) and a stem.  The noun \emph{ʃitere} `finger' has the prefix \emph{\esh{}i-} and the stem \emph{-tere}.  The prefix can change to show plurality, \emph{ma-tere}, as well as a few other features discussed later.

\fbox{Noun: prefix + root}

Based on word usage, agreement, and these prefixes, the nouns of Lwitaxo have been broken up into 19 different noun classes.  These noun classes match with general Bantu linguistics practices.  A noun word will fall into one of these classes and is normally easily recognizable by the prefix.  An important reason for these noun classes is agreement throughout the sentence: adjectives and other words agree with the nouns they correlate with in a sentence based off of the noun class of the word.

The bolded text mark the parts showing agreement:
\gll Soma\engma{}ga \emph{\textbf{\esh{}i}}tapu \emph{\textbf{\esh{}i}}lijaji ji\emph{\textbf{\esh{}i}}.
     {I am reading} book good this.
\glt `I am reading this good book.'
\glend

\gll \emph{\textbf{i}}ndzuxa \emph{\textbf{j}}ana je\emph{\textbf{j}}o \emph{\textbf{i}}litst\'{a}\engma{}gma l\i{}t\esh{}un\engma{}ga.
     snake young that {is eating}                       rat.
\glt That young snake is eating a rat.
\glend

Classes are generally paired into noun classes corresponding with singular and plural: \emph{mu-ndu} `person' is class 1 and \emph{\beta{}a-ndu} `people' is class 2.The most common singular/plural noun class pairings are: 1/2, 3/4, 5/6, 7/8, 9/10, 11/10, 12/13, and 20/4.  Some nouns exist with out a singular-plural distinction, such as liquids like \emph{matsi} `water' and \emph{makura} `oil' (both class 6) or emotions such as \emph{βujanzi} `happiness' (class 14).  There are also a few words which do not have a prefix, such as \emph{mama} `mother' and \emph{tata} `father'.  These prefixless nouns tend to be relationship words and all adjectives and other words agree with them as if they were class 1.

The locative noun classes (classes 16, 17, and 18) attach to other nouns see locative section below for more information about them.  The diminutive classes 12 and 13 (singular and plural) and the augmentative classes 20 and 4 (singular and plural) not only attach to other nouns, but also replace the pre-existing noun prefix.

\subsection{Noun Classes}\label{sec:nounclasses}

Noun classes 1 and 2 are composed of things that involve humans, such as occupations, words for relations, and descriptions such as child.  The prefix for class 1 is \emph{mu-} and for class 2 is \emph{\beta{}a-}.

\noindent\begin{tabular}{l l l l l l}
1 & \emph{mu-ndu} & \emph{mu-xana} & \emph{mw-ana} & \emph{mw-et\esh{}itsi} & \emph{mw-\i{}\beta{}uli} \\
  & `person' & `girl'           & `child'       & `teacher'            & `parent'\\
2  & \beta{}a-ndu & \beta{}a-xana & \beta{}-ana & \beta{}a-et\esh{}itsi & \beta{}-\i{}\beta{}uli\\
     & `people'          &  `girls'               & `children'      & `teachers'                   & `parents'\\
\end{tabular}\\
%%%%%%%%%%%%%%%%%%%%%%%%

Noun classes 3 and 4 have several nature items, such as the sun, moon, wind and trees; as well as several other nouns.  A more well defined connection of the semantic categories has yet to be found.  The prefix for class 3 is \emph{mu-} and is \emph{mi-} for class 4.

\noindent\begin{tabular}{l l l l l l}
3 & \emph{mu-k\'{a}ti} & \emph{mu-sala} & \emph{mu-eli} & \emph{mu-koje} & \emph{mu-xono} \\
  & `bread' & `tree'           & `month'       & `catfish'            & `arm/hand'\\
4  & \emph{mi-k\'{a}ti} & \emph{mi-sala} & \emph{mi-eli} & \emph{mi-koje} & \emph{mi-xono}\\
     & `breads' &  `trees'  & `months'      & `catfishes'  & `arms/hands'\\
\end{tabular}\\
%%%%%%%%%%%%%%%%%%%%%%%%

Classes 5 and 6 include an assortment of animals, including most birds, as well as a collection of plants and other items.  The prefix for class 5 is \emph{li-}, and for class 6 is \emph{ma-}.

\noindent\begin{tabular}{l l l l l l}
5 & \emph{li-kondi} & \emph{li-sitsa} & \emph{li-hondo} & \emph{lj-aro} & \emph{li-\palnas{}o\palnas{}i} \\
  & `sheep' & `week'     & `pumpkin'       & `boat'            & `bird' \\
6  & \emph{ma-kondi} & \emph{ma-sitsa} & \emph{ma-hondo} & \emph{m-aro} & \emph{ma-\palnas{}o\palnas{}i} \\
     & `sheep'  &  `weeks'  & `pumpkins'      & `boats'  & `birds'\\
\end{tabular}\\
%%%%%%%%%%%%%%%%%%%%%%%%

The prefix for class 7 is \emph{\esh{}i-} and for class 8 is \emph{\beta{}i-}.  There does not appear to be any distinct semantic categories tightly linked to classes 7 and 8.

\noindent\begin{tabular}{l l l l l l}
7 & \emph{\esh{}i-kala} &\emph{\esh{}i-kombe} & \emph{\esh{}i-tapu} & \emph{\esh{}i-mbuli} & \emph{\esh{}i-ret\esh{}elo} \\
  & `foot' & `cups'	     & `book'       & `goat'            & `village' \\
8  & \emph{\beta{}i-kala} & \emph{\beta{}i-kombe} & \emph{\beta{}i-tapu} & \emph{\beta{}i-mbuli} & \emph{\beta{}i-ret\esh{}elo} \\
     & `feet'  &  `cups'  & 			`books'      & `goats'  & `villages'\\
\end{tabular}\\
%%%%%%%%%%%%%%%%%%%%%%%%

Noun classes 9 and 10 include many animals as well as many modern day loan words, from English, Swahili, or other.  The class prefixes appear to be simply \emph{i-} and \emph{tsi}, but there is actually a nasal at the end of both prefixes, if the beginning of the root allows it: \emph{in-} and \emph{tsin}.  The nasal by default is `n', but will change place of articulation based off of the stem it attaches too. See the phonology section for more details.  Modern loan words do not preserve this final nasal of the prefix, as the example \emph{ikomputa} shows.

\noindent\begin{tabular}{l l l l l l}
9 & \emph{in-dzeku} & \emph{i-sutse} & \emph{i-mesa} & \emph{i-komputa} & \emph{i\engma{}-gulume} \\
  & `elephant' & `fish'	     & `table'       & `computer'            & `pig' \\
10  & \emph{tsin-dzeku} & \emph{tsi-sutse} & \emph{tsi-mesa} & \emph{tsi-komputa} & \emph{tsi\engma{}-gulume} \\
     & `elephants'  &  `fishes'  & `tables'      & `computers'  & `pigs'\\
\end{tabular}\\
%%%%%%%%%%%%%%%%%%%%%%%%

Noun class 11 has the prefix \emph{lu-}, and uses class 10, \emph{tsin-} as its plural class pair.  In this pair it much easier to see the nasal that exists at the end of the class 10 prefix, as class 11 does not have this nasal.
Noun class 11 has the prefix \emph{lu-}, and uses class 10, \emph{tsin-} as its plural class pair.  In this pair it much easier to see the nasal that exists at the end of the class 10 prefix, as class 11 does not have this nasal.

\noindent\begin{tabular}{l l l l l l}
11 & \emph{lu-ika} & \emph{lusala} & \emph{lu-pau} & \emph{lu-imbo} & \emph{lu-kaka} \\
  & `horn' & `stick'	     & `lumber'       & `song'            & `fence' \\
10  & \emph{tsin-zika} & \emph{tsi-sala} & \emph{tsim-bau} & \emph{tsin-imbo} & \emph{tsi\engma{}-gaka} \\
     & `horns'  &  `sticks'  & `lumbers'      & `songs'  & `fences'\\
\end{tabular}\\
%%%%%%%%%%%%%%%%%%%%%%%%

Classes 12 and 13 work differently then the previous noun classes.  Rather than having words which are  inate members of that noun class, words become this noun class when being used as the diminutive (dim.)\footnote{The diminutive is a form denoting small, cute, childish, etc.  For example, `doggy' or `kitty' would be a diminutive of dog and cat in English.}.  The class 12 prefix is \emph{xa-} and the class 13 prefix is \emph{ru-}.  These prefixes replace the original prefix of the word.

\noindent\begin{tabular}{l l l l l l}
Word & Plural & English & Cl. 12 & Cl. 13\\\hline
i-simbwa (Cl. 9) &  tsi-simbwa (Cl. 10) & `dog' & xa-simbwa & ru-simbwa\\
i\engma{}-gu\beta{}u (Cl. 9) & tsi\engma{}-gu\beta{}u (Cl. 10) & `hippo' & xa-ku\beta{}u & ru-ku\beta{}u\\
li-lesi (Cl. 5)  & ma-lesi (Cl. 6) & `cloud' & xa-lesi & ru-lesi \\
\esh{}i-kombe (Cl. 7) & \beta{}i-kombe (Cl. 8) & `cup' & xa-kombe & ru-kombe\\
\end{tabular}\\
%%%%%%%%%%%%%%%%%%%%%%%%

The majority of class 14 nouns deal with emotions and state of being in time (like childhood) and therefore do not exist in the plural.  The few class 14 nouns that do have a plural use class 4 as the plural class.  Class 14's prefix is \emph{\beta{}u-} and class 4's is \emph{mi-}.

\noindent\begin{tabular}{l l l l l l}
14 & \emph{\beta{}u-hiendela} & \emph{\beta{}w-eni} & \emph{\beta{}u-janzi} & \emph{\beta{}u-\'{a}na} & \emph{\beta{}u-ija} \\
  & `adulthood' & `forehead'	     & `happiness'       & `childhood'            & `body hair' \\
4  & --- & \emph{mj-eni} & --- & --- & --- \\
     & ---  &  `foreheads'  & ---      & ---  & ---\\
\end{tabular}\\
%%%%%%%%%%%%%%%%%%%%%%%%

Noun class 15 does not have a singular/plural pair.  It is used for creating nominal forms of verbs.  These are often translated as the infinite forms (e.g. `to walk') or the participle form (`walking').  The prefix is \emph{xu-}.  Refer to section \ref{sec:infinitives} for more information.

\noindent\begin{tabular}{l l l l l l}
15 & \emph{xu-\palnas{}ola} & \emph{xu-hun\engma{}gma} & \emph{xu-kona} & \emph{xu-\beta{}a} & \emph{xu-mila} \\
  & `to find' & `to drink'	     & `to sleep'       & `to be'            & `to swallow' \\
\end{tabular}\\

%%%%%%%%%%%%%%%%%%%%%%%%%

Noun classes 16, 17, and 18 are all locative cases.  The locative case is added to the already existing noun case.  A brief discussion of them is below, though more data needs collected to get acquire a better picture of how the locatives function.

Class 16, prefix \emph{ha-}, is used to emphasize location on a surface or motion towards being on top of a surface, such as \emph{lipata lili \emph{\textbf{ha}}kofija} `the duck is \emph{on top of} the hat' where using the generic `in' would cause confusion as to whether the duck is in or on the hat.  It can also be used to show `near', such as \emph{ndi \emph{\textbf{ha}}litu:ka,} `I'm near the shop.'

Class 17 is the most common locative case and can generally be translated as `in' or `to' and is used in instances where there doesn't need to be an emphasis.  For example, the English sentence, `They were in a boat on the lake' would be in Lwitaxo: \emph{\beta{}a:li \emph{\textbf{xw}}aro \emph{\textbf{xu}}\palnas{}anza.}  The instances where class 18, prefix \emph{mu-}, is used is unclear.

\subsection{Word order in the noun phrase} \label{sec:nounwordorder}
The segments of the noun phrase follow this ordering:\\
\mbox{noun possessive number adjective demonstrative} or \fbox{N POS \# ADJ DEM}\\
Of these, only the noun is required to be present.

A few examples of noun phrases using different components of the noun phrase:
\begin{wrdex}
\item \glll βajaji βand\ezh{}e βaβaka βuβujandzi jaβa
boys my three happy these
N POS \# ADJ DEM
\glt `these, my three happy boys'
\glend
\item \glll indzu i\engma{}gali jiji
house big this
N ADJ DEM
\glt `this big house'
\glend
\item \glll makondi koko karanu makali
sheep your five big
N POS \# ADJ
\glt ` your five big sheep'
\glend
\end{wrdex}

\section{And}
There are two words for `and': \emph{nende} and \emph{ni}.  Both are equivalent in meaning and in most situations inter-changeable.  \emph{nende} simply goes between the two words being connected such as, \emph{lukanu nende \esh{}itapu} `the folktale and the book.'  \emph{ni} is a little bit more complicated, as it often replicates the vowel following it, such as \emph{lukanu ni \esh{}itapu} `the folktale and the book' if reversed in order becomes \emph{\esh{}itapu nu lukanu} `the book and folktale'.  Other examples: \emph{isimba nu mwana wajo} `the lion and its cub', \emph{isimba na \beta{}ana \beta{}ajo} `the lion and its cubs.'

\emph{ni} and \emph{nende} `and' are used to show most connections between nouns or verbs.  Such as possession, which is shown with the possessor as the subject, the verb to be, the word and, and then the possessed: \emph{ndi ni imbuli} `I have a goat,' \emph{ndi na maremwa} `I have bananas.'

\section{Interrogative words}
To ask about an object, the word for what is \emph{ni\esh{}i}.
\begin{wrdex} \item \emph{ni\esh{}i je\esh{}o}, when pointing at a pencil would translate to, `What's that?' \end{wrdex}

This \emph{\esh{}i} is also used in conjunction with nouns to mean `which.'  This is the preferred method of asking about time, if the unit of time is predictable.
\begin{wrdex} \item \emph{witsa America muhika \esh{}i} `In which year did you come to America?'
\item \emph{uhuja lituxu \esh{}i} `What day did you leave?' \end{wrdex}

For people, the word for who is \emph{ni\beta{}i}.
\begin{wrdex} \item \emph{ni\beta{}i ojo} when pointing at a person translates to `Who is that?'  \item Equally acceptable for this question is \emph{ni\beta{}ina ojo}. \end{wrdex}

The question why is asked with the word, \emph{\esh{}it\esh{}ila\esh{}i}.
\begin{wrdex} \item \emph{\esh{}it\esh{}ila\esh{}i ulila\engma{}ga} `Why are you crying?' \end{wrdex}

The question when is asked with the word \emph{sa:\esh{}i}.  Using this form rather than the unit of time plus the \emph{\esh{}i} mentioned above shows that whether the event happened minutes, hours, days, or years ago.
\begin{wrdex} \item \emph{ulili \esh{}ixulia sa:\esh{}i} `When did you eat the food?' \end{wrdex}