\chapter{Phonetics and Phonology}
\section{Sounds of Lwitaxo}
\subsection{Consonants}\label{sec:consonants}

\begin{minipage}{\linewidth}
\begin{tabular}{|l | c | c | c | c | c | c|}
\hline & Bilabial & Alveolar & P-alveo & Palatal & Velar & Glottal \\ \hline
Plosive & p (b)\footnote{Letters in parenthesis only exist as allophones} & t (d) & & & k (g) & \\\hline
Nasal & m & n & & \palnas{} & \engma{} & \\\hline
Affricates &  & ts (dz) & t\esh{} (d\ezh{}) & & & \\ \hline
Fricatives & \beta{} f & s & \esh{} & & x & h\\ \hline
Flap/Trill & &l \alvlatflap{}\footnote{it was discovered on the last day of research that l and \alvlatflap{} are possibly distinct phonemes, where as no difference was previously heard.  Due to this, the data in this grammar does not yet show this distinction.} r & & & & \\\hline
Glide & (w) & & & (j) & & \\ \hline
\end{tabular}
\end{minipage}

\subsection{Vowels}\label{sec:vowels}

Lwitaxo has a 5 vowel system.  There are high and mid front vowels, and high and mid back vowels.  The low vowel does not have a front/back distinction.

\begin{center}
\begin{vowel}[simple,three]
 \putcvowel{i}{1}
\putcvowel{\epsilon}{2}
\putcvowel{a}{5}
\putcvowel{\openo}{7}
\putcvowel{u}{8}
\end{vowel}
\end{center}

\begin{description}
\item[/i/] The high, unrounded, front vowel includes a large area of vowel space.  The actual realization of this sound when spoken includes everything from [i] to the upper limits of [e].  This means that the vowel written as `i' in this grammar might be pronounced as [i] in `tree', to [\openi] in `pit', to something that sounds a lot like [e] in `mate'.

\item[/\epsilon{}/] The mid, unrounded, front vowel is realized as [\epsilon{}], as in the English word `bed'.  The sound is transcribed with the letter `e' in this grammar.

\item[/u/]  The high, rounded, back vowel covers a very similar territory height-wise as /i/ does.  It is realized as [u] (as in English `moo') and also as [\openu{}] (as in English `lull').

\item[/\openo{}/] The mid, rounded back vowel is realized as [\openo] similar the English word `bog'.  In this grammar it is written as `o'.

\item[/a/] The /a/ is pronounced as a low back vowel.  It is the only low vowel in the vowel system.
\end{description}

\subsection{Tone}

Lwitaxo has two phonemic tones: high and low.  The low tone is the more neutral, and could also be referred to as `not-high.'  The tone is not fixed onto a syllable, but may move to a different syllable when a word is put in a sentence or a verb is conjugated

\begin{wrdex}
\item \emph{nimb\'a} `I sing' compared with \emph{nimba} `I sing.'
\item \emph{í\engma{}g\'u\beta{}u} `dress' compared with \emph{í\engma{}gu\beta{}u} `hippo'
\item \emph{ísímba} `lion compared with \emph{isimba} `hut for an unmarried man'
\item \emph{xulol\'a\engma{}ga} `I see you' but \emph{axulola\engma{}ga} `he/she sees you'
\item \emph{mb\'aja} `I play' but \emph{mbaja katí} `I play the game kati'
\end{wrdex}

\subsection{Length}

Vowel length is distinctive in Lwitaxo.  There appear to be 2 phonemic vowel lengths: long and short.  There do not seem to be many examples where the word differentiates only by vowel length, except for the the formation of the recent past perfective and the short form of the recent past imperfective.  In many cases where a vowel is deleted, there is a vowel lengthing as compensation.  Long vowels are notated with `:' following the vowel.

\begin{wrdex}
\item \emph{nd\ezh{}endi} `I walked (recent)' compared with \emph{nd\ezh{}e:ndi} `I was walking (recent)'
\item \emph{\engma{}wele} `I drank (recent)' compared with \emph{\engma{}we:le} `I was drinking (recent)'
\end{wrdex}

\section{Syllable Structure}
The possible syllable structure is (N)(C)(G)V.  The basic syllable is CV, and there are no codas allowed.  The only kinds of complex onsets allowed is for the consonant to be preceded by a nasal or succeeded by a glide. It is possible for the syllable to have both the nasal and the glide. The consonant without the onset, being only a vowel, seems to occur only in class 9 nouns\footnote{for a discussion of the noun class system see section \ref{sec:nounclasses}}, as well as 2nd and 3rd person singular verbs.

\fbox{(N)(C)(G)V}\\
% Table generated by Excel2LaTeX from sheet 'Sheet2'
\begin{tabular}{lll}
\addlinespace
\toprule
CV    & {\it \textbf{mi.hi.ka}} & `years' \\
V     & {\it \textbf{i}.tʃi.li.ʃi} & `bull' \\
NCV   & {\it tsi.\textbf{ndu}.ma} & `yams' \\
CGV   & {\it \textbf{xwi}.βi.li.la} & `to forget' \\
NCGV  & {\it i.si.\textbf{mbwa}} & `dog' \\
\bottomrule
\end{tabular}

\section{Phonological Rules}

\subsection{The nasals}

The nasals m,n,\palnas{}, and \engma{} appear to be distinctive sounds in Lwitaxo.

% Table generated by Excel2LaTeX from sheet 'Sheet2'
\begin{tabular}{ll}
\addlinespace
\toprule
{\it \textbf{mu}rio } & `thank you' \\
{\it  βwa\textbf{na} } & `childhood' \\
{\it  i\textbf{ɲi}ni\textbf{ɲi}ni } & `star' \\
{\it litʃu\textbf{ŋwa}} & `orange' \\
\bottomrule
\end{tabular}

Despite being distinctive, the nasals will assimilate to the place of articulation of a following consonant and neutralize the distinction of place of articulation in the nasal, as well as also causing the following consonant to voice.

  [nasal] \becomes [\alpha{} place] \inthe \environ [\alpha{} place]

  [+stop] \becomes [+voice] \inthe [+nasal] \environ

The sounds [h] and [\beta{}] also change when preceded by a nasal.  The \beta{} becomes a stop, [b], and the [h] acts like a [p]: voicing and becoming a stop [b] which causes the nasal to be realized as an [m].\\
\noindent\begin{minipage}{\linewidth}
\resizebox{\textwidth}{!}{ %
% Table generated by Excel2LaTeX from sheet 'Sheet2'
\noindent\begin{tabular}{llllll}
\addlinespace
\toprule
{\bf n + } & {\bf Results in} & {\bf Example} &       & {\bf Without n} &  \\
\midrule
{\bf n + p} & mb    & tsimbau & `woods'\footnote{as in different types of woods} & lupau & `wood' \\
{\bf n + t} & nd    & indana & `infant' & xatana &  `little infant' \\
{\bf n + k} & \engma{}g & ŋgulaŋga  & `I'm buying' & xukula & `to buy' \\
{\bf n + \beta{}} & mb    & mbimbitsa & `I'm boiling' & xu\beta{}imbitsa & `to boil' \\
{\bf n + h} & mb    & mbuja\engma{}ga & `I'm going home' & xuhuja & `to go home' \\
{\bf n + t\esh{}} & nd\ezh{} & ndʒenda\engma{}ga & `I'm walking' & xuts\esh{}enda & `to walk' \\
{\bf n + ts} & ndz   & indzu & `house' & xatsu & `little house' \\
{\bf n + j} & ndz   & indzuxa & `snake' & xajuxa & `little snake' \\
{\bf n + l} & nd & ndi & `I am' & ali & `he/she is' \\
{\bf n + l}\footnote{The reason for the two different results of `n+l' is unclear at this time: it may be a result of the two different l's: [l] and [\alvlatflap{}]} & n & nomaloma & `I speak' & ulomaloma & `you speak'\\
{\bf n + r} & nd & ndandula\engma{}ga & `I'm tearing' & xurandula & `to tear'\\
\bottomrule
\end{tabular}}
\normalsize
\end{minipage}

\subsection{Deletion of stop after a nasal}
When there are two syllables in a row which both have the nasal+consonant onset, then the consonant of the first of these syllables is deleted.

\fbox{NC...NC \becomes N...NC}

Example: \emph{i\engma{}gombe} is pronounced [i\engma{}ombe], however the diminutive form of cow, \emph{xakombe}, shows an underlying k (expected to turn into a g after the nasal of the class 9 prefix, see section \ref{sec:nounclasses}) which is not present in the pronunciation of the word.  The same situation occurs with [\engma{}] in \emph{iŋumbu} `an article of clothing'.

\subsection{Palatalization}
When velars are followed by front vowels, they palatalize.  The following chart shows what sounds palatalize when followed by a front vowel ([i] or [e]) and what they palatalize to.

\noindent\begin{minipage}{\linewidth}
\resizebox{\textwidth}{!}{ %
% Table generated by Excel2LaTeX from sheet 'Sheet1'
\begin{tabular}{rlll}
\addlinespace
\toprule
{\bf sound} & {\bf palatalized} & {\bf example} &  \\
\midrule
k     & t\esh{} & \emph{xwi:luxaka} `to run & \emph{nax\'iluxat\esh{}e} `I will run (remote)' \\
\engma{}g & \emph{nd\ezh{}} &  \emph{-a\engma{}ga} impf marker & \emph{at\esh{}e:ndiend\ezh{}e} `he/she was walking (near)' \\
\bottomrule
\end{tabular}}
\normalsize
\end{minipage}

\subsection{Vowel interactions}\label{sec:arules}
When the vowel [a] is the first vowel in an underlying two vowel sequence ([aV]), then the [a] deletes, unless the second vowel is also an [a], then the vowels combine to form a long vowel.\\
\fbox{[ai,au,ao,ae] \becomes [i,u,o,e]}\\
\fbox{[a] +[a] \becomes [a:]}

\begin{wrdex}
\item u+la+imb+a: \emph{ulimba} `you will sing (near future)' 
\item a+la+iluxak+a: \emph{alilux\'aka} `he/she will run (near future)' 
\item \beta{}a+uli: \emph{\beta{}uli} `they are' 
\item \beta{}a+\'e\palnas{}+a: \emph{\beta{}\'e:\palnas{}a} `they wanted (near past)
\item βa+anafúndzi: \emph{βa:nafúndzi} `students' 
\end{wrdex}

\subsection{Glide formation} \label{sec:glides}
When a high vowel precedes another vowel, the high vowel becomes a glide.  [i] becomes [j], and [u] becomes [w].  For example, the class 1 prefix `mu' when attached to a root that begins with a vowel, becomes `mw' as in mwana `child'.  It appears that if the high vowel is also high tone and the following vowel is low toned, then the glide formation does not take place or might be optional, more research is needed on this.

\begin{featmatrix}+vowel\\+high\end{featmatrix} \becomes [+glide] \inthe \environ V

\subsection{Vowel Harmony}\label{sec:vowelharmony}
Vowel harmony can often affect what vowel occurs at the end of a word.  It is not universally applied and not all of the instances where vowel harmony occurs are certain.

For example, the final vowel of the perfective near past is [e] unless the preceding vowel is high (i or u), then the final vowel becomes [i].  For example, ndili `I just ate' versus \engma{}wele `I just drank.'